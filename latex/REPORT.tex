\documentclass[12pt,a4paper]{article}
\usepackage[utf8]{inputenc}
\usepackage{graphicx}
\usepackage{float}
\usepackage[hidelinks]{hyperref}
\usepackage{geometry}
\usepackage{siunitx}
\geometry{margin=1in}

% Define placeholder text command
\newcommand{\placeholder}[1]{%
Lorem ipsum dolor sit amet, consectetur adipiscing elit. Sed do eiusmod tempor incididunt ut labore et dolore
    magna aliqua. Ut enim ad minim veniam, quis nostrud exercitation ullamco laboris nisi ut aliquip ex ea commodo consequat. Duis aute irure dolor in reprehenderit in voluptate velit esse cillum dolore eu fugiat nulla pariatur. Excepteur sint occaecat cupidatat non proident, sunt in culpa qui officia deserunt mollit anim id est laborum.
\par\noindent Sed ut perspiciatis unde omnis iste natus error sit voluptatem accusantium doloremque laudantium, totam rem aperiam, eaque ipsa quae ab illo inventore veritatis et quasi architecto beatae vitae dicta sunt explicabo.
}

\title{Wetris: CPE 301 Final Project}
\author{Team Wetris (Team 42)}
\date{December 12, 2025}

\begin{document}

\maketitle
\tableofcontents
\newpage

\section{Project Description}
Wetris is a suspenseful and slightly nerve-racking spin-off of Tetris, the classic block-stacking game. The player races to clear enough lines to reach the target score before their stack reaches the top, and if they fail, they get splashed with water! To make it even better, the spring powered water gun is very loud and is a garenteed jump scare even if you know it's coming. It makes for a great party game, and would go well as an attraction in an arcade.

\section{Cost Analysis for this First Time Build}
\label{sec:cost-analysis}

This cost analysis includes tools that will not need to be repurchased for future builds, such as the multimeter and crimping tool. These tools were necessary for the successful completion of this first build, and their costs are included to provide a comprehensive overview of the initial investment required for the Wetris project. A production version of Wetris could be built for significantly less, since the tools are one time purchases, and many of the components were purchased in higher quantities than necessary and the cost for those components could be spread across multiple builds. The largest cost would still be the 3D printed casing, but even that could be reduced with bulk printing or alternative manufacturing methods such as injection molding. A cost analysis for a production version is not included here, but may later be developed if there is interest in pursuing that avenue.

\begin{table}[H]
\centering
\small
\begin{tabular}{|l|r|p{5cm}|}
\hline
\textbf{Material} & \textbf{Total Price} & \textbf{Notes} \\ \hline
3D printed casing & \$263.52 & UNR LIBRARIES TRANSACTION 201937 DLM \\ \hline
Arduino ATmega 2560 & \$54.02 & Amazon \\ \hline
Multimeter & \$34.63 & Amazon \\ \hline
Crimping tool & \$34.63 & Amazon \\ \hline
LCD Display & \$22.72 & Amazon (Jorge also bought us a second one for testing) \\ \hline
Wire stripper & \$20.56 & ACE hardware \\ \hline
Door lock actuator & \$17.05 & Amazon \\ \hline
Relay Module & \$14.42 & Amazon \\ \hline
Syringes & \$14.06 & Amazon \\ \hline
12V power adapter & \$14.06 & Amazon \\ \hline
Extension cord & \$12.98 & Amazon \\ \hline
Caulk gun & \$10.82 & Newborn DC012 from ACE hardware \\ \hline
Pin connector for actuator & \$10.27 & Amazon \\ \hline
Electrical tape & \$9.30 & ACE hardware \\ \hline
Wire & \$8.66 & ACE hardware \\ \hline
Inline fuse holder & \$7.13 & ACE hardware \\ \hline
Buzzers & \$6.49 & Amazon \\ \hline
Sprinkler head & \$5.40 & ACE hardware \\ \hline
Fuses & \$5.40 & Amazon \\ \hline
Barrel jack splitter & \$4.30 & Amazon \\ \hline
Misc. we on hand & \$0.00 & Jumper wires, buttons, tools, resistors, breadboard, 9V battery, USB cable, opto-couplers etc. \\ \hline
Epoxy & \$0.00 & Took a little from another project of Gabe's \\ \hline
\hline
\multicolumn{1}{|r|}{\textbf{Grand Total:}} & \textbf{\$570.43} & \\ \hline
\end{tabular}
\caption{Component cost breakdown for Wetris system}
\label{tab:components}
\end{table}

\section{Component Details}
This section provides detailed information about each major component used in the Wetris system, including their specifications and roles within the project.

\subsection{Arduino Mega 2560}
Central microcontroller unit (MCU) responsible for managing game logic, user inputs, and controlling peripherals.
Full technical specifications can be found at: \\
\url{https://cdn.robotshop.com/media/a/ard/rb-ard-33/pdf/arduinomega2560datasheet.pdf}
for the development board and \\
\url{https://www.alldatasheet.com/datasheet-pdf/download/107092/ATMEL/ATMEGA2560.html}
for the microcontroller itself.

\subsection{LCD Display (ST7796S controller)}
Hosyond 4.0" TN Capacitive Touch Display\\
Display Controller: ST7796S driver IC\\
Touch Controller: FT6336U capacitive touch IC (not used in project)\\
Size: 4.0" diagonal, 320x480 resolution\\
Physical Dimensions: 60.88x108.0x14.80mm (WxHxD)\\
Interface: 4-wire SPI (display) + I2C (touch)\\
Power: 5V/3.3V compatible, 0.5W consumption (We used 5V)\\
Connectivity: 14P 2.54mm header + FPC connector\\
Additional: Micro SD slot, LED backlight (300 cd/m²) (Both unused in project)

\subsection{Door Lock Actuator}
The door lock actuator used is intended for a Dodge Grand Caravan, and operates on 12V DC and draws 2A. In order to reverse the direction of the actuator, the polarity of the voltage applied to it must be reversed. It is energy efficient, only drawing current when moving, and has a built-in limit switch to stop current flow when fully extended or retracted. Full specifications could not be found past what the amazon page provided. Pins had to be experementally probed to determine their propper function. Two pins still have unknown function and are left unconnected. They are possibly for feedback.
More information can be found at: \\
\url{https://www.amazon.com/dp/B0C43BNSQM?ref_=ppx_hzsearch_conn_dt_b_fed_asin_title_3&th=1}

\subsection{Fuse}
Standard 3A automotive blade fuse used to protect the water gun actuator circuit from overcurrent conditions.

\subsection{Potentiometer}
Standard 10k ohm rotary potentiometer used to adjust game speed and music tempo.

\subsection{Buzzers, Resistors, and Capacitors}
Standard piezoelectric buzzers used for audio output, along with \qty{100}{\ohm} resistors and \qty{10}{\micro\farad} capacitors for signal conditioning and circuit protection.

\subsection{Relay Module}
Forward and Reverse Relay Module, 12V 10A Pre-Wired with LED Light, for Motor/Linear Actuator, Reversing Relay Module.
More information can be found at: \\
\url{https://www.amazon.com/Weasch-Forward-Pre-Wired-Actuator-Reversing/dp/B0FJQV4316?pd_rd_w=4c3sO&content-id=amzn1.sym.97c08ff8-4f39-4f9e-b360-8dbdbec34ee6&pf_rd_p=97c08ff8-4f39-4f9e-b360-8dbdbec34ee6&pf_rd_r=YHHMAAYABW9QQTN9YD5D&pd_rd_wg=pirQI&pd_rd_r=114b95df-9ba3-4874-ba37-76236030403c&pd_rd_i=B0FJQV4316&psc=1&ref_=pd_basp_d_rpt_ba_s_1_t}

\subsection{Optocouplers}
4N35 optocouplers with a foreward voltage of \qty{1.18}{\volt} and a maximum collector-emitter voltage of \qty{30}{\volt} used to isolate the high current 12V actuator control circuit from the low voltage 5V logic of the Arduino.
More information can be found at: \\
\url{https://www.alldatasheet.com/datasheet-pdf/view/98359/FAIRCHILD/4N35.html}


\section{System Overview}

For this project we designed a series of peripherals to interface with the microcontroller in the Arduino Mega 2560 development board, and connected them in a 3D printed casing. The design is highly modular, with each peripheral being controlled by a library, and being handled from a high level main loop. See code and schematic for most in depth explaination. The peripherals are:
\begin{itemize}
\item The Tetris module, consisting only of an LCD, three buttons, and the software to run tetris game logic
\item The music module, with three buzzers playing the tetris theme song (Korobeiniki)
\item The water gun module, which consists of a syringe, which is pushed by a spring powered plunger let back by a door lock actuator controlled by a relay module, which is in turn controlled by opto-couplers.
\item The potentiometer module, which is used to adjust the speed of the music module and the Tetris module.
\end{itemize}


\subsection{Hardware Architecture}
The following table outlines the microcontroller resources allocated to each component of the Wetris system, and details the connections used:

\begin{table}[ht]
\centering
\begin{tabular}{|l|l|l|}
\hline
\textbf{Reserved Component} & \textbf{Name of User} & \textbf{Purpose} \\ \hline
OC1A(pin11)+ Timer1 & Gabe & PWM signal for music \\ \hline
OC3A(pin5) + Timer3 & Gabe & PWM signal for music \\ \hline
OC4A(pin6) + Timer4 & Gabe & PWM signal for music \\ \hline
Button 1(pin13) & Roman+Jorge+Gabe & Left Movement \\ \hline
Button 2(pin12) & Roman+Jorge+Gabe & Right Movement \\ \hline
Button 3(pin10) & Roman+Jorge+Gabe & Rotation Movement \\ \hline
LCD(pins 7-9) & Roman & LCD connections \\ \hline
LCD(Dig. Pins50-52) & Roman & LCD connections \\ \hline
LCD LED(pin2) & Roman & Backlight control \\ \hline
FWD Pin (22) & Son+Gabe & Drive actuator forward \\ \hline
REV Pin (23) & Son+Gabe & Drive actuator reverse \\ \hline
\end{tabular}
\caption{Hardware resource allocation for Wetris system}
\label{tab:hardware}
\end{table}

\subsubsection{Tetris Module}
In order to reach the back of the casing where the MCU is mounted from the LCD and buttons in the front of the casing, we used 16 AWG wire to connect the buttons and dupont wires linked end to end to connect the LCD. The LCD pin connections in detail are as follows:
\begin{itemize}
    \item Pin 2  $\rightarrow$ LED pin on the LCD
    \item Pin 4  $\rightarrow$ Rotate Piece Button
    \item Pin 8  $\rightarrow$ RESET pin on LCD
    \item Pin 9  $\rightarrow$ CD pin on LCD
    \item Pin 10 $\rightarrow$ CS pin on LCD
    \item Pin 12 $\rightarrow$ Right Button
    \item Pin 13 $\rightarrow$ Left Button
    \item Pin 50 $\rightarrow$ SDO
    \item Pin 51 $\rightarrow$ SDI
    \item Pin 52 $\rightarrow$ SCK
\end{itemize}

\subsubsection{Music Module}
The music module uses output compare pins OC1A (pin 11), OC3A (pin 5), and OC4A (pin 6) to output PWM signals to three separate piezo buzzers. Each buzzer is connected in series with a 100 ohm resistor to limit current and a \qty{10}{\micro\farad} capacitor to smooth the signal. The code directly loads frequency and duration values into memory and iterates through them to play the song. This method is highly memory intensive, but allows for accurate timing and frequency control. The music was converted from midi format to c style arrays using a custom python script found in the repository.

\subsubsection{Water Gun Module}
In order to isolate the high current 12V power supply used to drive the water gun actuator from the 5V logic of the Arduino, we used a relay module controlled by opto-couplers. The relay module is powered by the 12V supply, and the opto-couplers are powered by the 5V supply from the Arduino. The opto-couplers are connected to digital pins 22 and 23 on the Arduino, which control the forward and reverse movement of the actuator respectively. When the game is over and the safety threshold has not been reached, the Arduino activates the relay in reverse to trigger the water gun mechanism. The foreward function is not used in normal operation, but is included for completeness. Very importantly, an inline fuse is included in the 12V power line to protect against short circuits.

\subsubsection{Potentiometer Module}
The potentiometer is our most simple peripheral, being connected to the 5V and GND pins on the Arduino, with the wiper connected to analog pin A0. The potentiometer is used to adjust the speed of the game and music modules, allowing the player to customize their experience. Also allowing their friends to set it to an impossible speed in the middle of a game for extra fun.

\subsection{Software Architecture}
The software is structured in a modular fashion, with each peripheral being handled by its own library. The main loop coordinates the interactions between the different modules, ensuring smooth gameplay and timely responses to user inputs. All modules had to be engineered to be non-blocking and work together. This presented a difficult engineering challenge that had to be overcome with many optimisations and plan-B methodologies. For example, the LCD tetris module had to render only when the game state changed, otherwise the music module would stutter as the updates took too much time. Clearing the screen with the LCD library took too long, so we had to implement our own function to only update the parts of the screen that changed, and now we only use the clearing function when absolutely necessary, such as at the start of the game or when a row is cleared.

\subsection{Power}
The Wetris system needs two power sources to operate correctly. The first is power to the Arduino, which is supplied through the barrel jack port via a 9V battery (the 9V is internally regulated to 5V by the Arduino). The second power source is a 12V power supply which powers the water gun actuator via a relay module. The relay module is controlled by the Arduino through opto-couplers to isolate the two power sources.

\clearpage

\section{Images}

These images showcase the physical implementation of the Wetris system, including the circuit and various stages of gameplay.

\begin{figure}[H]
\centering
\includegraphics[width=0.8\textwidth]{photos/Circuit.jpeg}
\caption{Physical circuit implementation of Wetris system shown from above}
\label{fig:circuit}
\end{figure}

\begin{figure}[H]
\centering
\includegraphics[width=0.7\textwidth]{photos/Normal Gameplay.jpeg}
\caption{Normal gameplay showing Tetris blocks falling and stacking}
\label{fig:gameplay}
\end{figure}

\begin{figure}[H]
\centering
\includegraphics[width=0.7\textwidth]{photos/Moving and Rotating.jpeg}
\caption{Player moving and rotating blocks during gameplay}
\label{fig:moving}
\end{figure}

\begin{figure}[H]
\centering
\includegraphics[width=0.7\textwidth]{photos/Block Clear.jpeg}
\caption{Clearing the screen before block clearing animation when a row is completed}
\label{fig:clear}
\end{figure}

\begin{figure}[H]
\centering
\includegraphics[width=0.7\textwidth]{photos/point score.jpeg}
\caption{Player scoring a point by completing a row}
\label{fig:score}
\end{figure}

\begin{figure}[H]
\centering
\includegraphics[width=0.7\textwidth]{photos/win.jpeg}
\caption{Victory screen displayed when safety threshold is reached after game over}
\label{fig:win}
\end{figure}

\begin{figure}[H]
\centering
\includegraphics[width=0.7\textwidth]{photos/loss.jpeg}
\caption{Game over screen displayed when player loses}
\label{fig:loss}
\end{figure}

\begin{figure}[H]
\centering
\includegraphics[width=0.7\textwidth]{photos/You Needed A Wash.jpeg}
\caption{Game over screen with debug statements shown on serial monitor}
\label{fig:wash}
\end{figure}

\begin{figure}[H]
\centering
\includegraphics[width=0.7\textwidth]{photos/Relay On.jpeg}
\caption{Relay module in activated state controlling water gun actuator}
\label{fig:relay-on}
\end{figure}

\begin{figure}[H]
\centering
\includegraphics[width=0.7\textwidth]{photos/relay off.jpeg}
\caption{Relay module in deactivated state}
\label{fig:relay-off}
\end{figure}

\begin{figure}[H]
\centering
\includegraphics[width=0.7\textwidth]{photos/intheface.jpeg}
\caption{Player getting wet from the water gun after losing the game}
\label{fig:intheface}
\end{figure}

\section{Schematic Diagrams}

Below is a schematic diagram illustrating the electrical connections and components used in the Wetris system.
There is a high level diagram that shows how the peripherals are connected, and a more detailed schematic for each peripheral.

\begin{figure}[H]
\centering
\fbox{\parbox{0.8\textwidth}{\centering\vspace{2in}Schematic Diagram Placeholder\vspace{2in}}}
\caption{High level electrical schematic of Wetris system}
\label{fig:schematic-high-level}
\end{figure}

\begin{figure}[H]
\centering
\fbox{\parbox{0.8\textwidth}{\centering\vspace{2in}Schematic Diagram Placeholder\vspace{2in}}}
\caption{Electrical schematic of water gun module}
\label{fig:schematic-water-gun}
\end{figure}

\begin{figure}[H]
\centering
\fbox{\parbox{0.8\textwidth}{\centering\vspace{2in}Schematic Diagram Placeholder\vspace{2in}}}
\caption{Electrical schematic of music module}
\label{fig:schematic-music}
\end{figure}

\begin{figure}[H]
\centering
\fbox{\parbox{0.8\textwidth}{\centering\vspace{2in}Schematic Diagram Placeholder\vspace{2in}}}
\caption{Electrical schematic of LCD and button module}
\label{fig:schematic-lcd-buttons}
\end{figure}

\begin{figure}[H]
\centering
\fbox{\parbox{0.8\textwidth}{\centering\vspace{2in}Schematic Diagram Placeholder\vspace{2in}}}
\caption{Electrical schematic of potentiometer module}
\label{fig:schematic-potentiometer}
\end{figure}

\begin{figure}[H]
\centering
\includegraphics[width=0.8\textwidth]{photos/Casing2.0.pdf}
\caption{Solidworks technical drawing of 3D printed casing}
\label{fig:schematic-casing-drawing}
\end{figure}

\begin{figure}[H]
\centering
\includegraphics[width=0.8\textwidth]{photos/sld1.png}
\caption{Solidworks model of 3D printed casing}
\label{fig:schematic-casing-solidworks-1}
\end{figure}

\begin{figure}[H]
\centering
\includegraphics[width=0.8\textwidth]{photos/sld2.png}
\caption{Screenshot of Solidworks model of 3D printed casing from a roughly isometric angle}
\label{fig:schematic-casing-solidworks-2}
\end{figure}

\begin{figure}[H]
\centering
\includegraphics[width=0.8\textwidth]{photos/sld3.png}
\caption{Second screenshot from inside Solidworks from a different angle}
\label{fig:schematic-casing-solidworks-3}
\end{figure}

\begin{figure}[H]
\centering
\includegraphics[width=0.8\textwidth]{photos/sld4.png}
\caption{Fourth screenshot from inside Solidworks from a different angle}
\label{fig:schematic-casing-solidworks-4}
\end{figure}

\section{Links}

\subsection{Repository}
GitHub Repository: \\
\url{https://github.com/ACertainArchangel/Wetris-CPE-301-Final-Project}

\subsection{Documentation}
Additional documentation and resources can be found in the project repository.

\subsection{Demo Videos}
Link to demo videos: \\
\url{https://drive.google.com/drive/folders/1Q4Bg4BSVfFWwpvJh6P310yMYkDlfBsM4?usp=sharing}

\subsection{3D Printing Files}
Link to stereolithography (STL) 3D printing files: \\
\url{https://drive.google.com/drive/folders/11z260TUOtaoRf2S6610Rd5JktC3WCf8y?usp=sharing}

\section{Environmental Impact as Detailed in CPE 301 Grading Rubric}

\subsection{Energy Efficiency}
Wetris is designed to be energy efficient by utilizing low-power components to minimize power consumption during operation. The relay module and door lock actuator are only powered when necessary, reducing overall energy usage.

\subsection{Design Safety}
Wetris incorporates several safety features to ensure user safety during operation. The use of optocouplers isolates high voltage components from the low voltage control circuitry, preventing electrical hazards. Additionally, an inline fuse protects against overcurrent conditions.

\subsection{Affordability}
Considerations were given to potential cost reductions for mass production, such as bulk purchasing of components and exploring alternative manufacturing methods for the casing such as injection molding to make Wetris more affordable for consumers (see \hyperref[sec:cost-analysis]{\textbf{cost analysis section}}).

\subsection{Sustainability}
Wetris is designed with sustainability in mind by selecting components that are durable and have a long lifespan, reducing the need for frequent replacements. The 3D printed casing is made from PLA, which is derived from plant matter, to minimize environmental impact.

\subsection{Accessibility}
Wetris is designed to be accessible to a wide range of users by providing simple and intuitive controls. The use of large buttons and a clear LCD display ensures that users of varying abilities can easily interact with the system.

\section{Our Team}

\subsection{Gabriel Jordaan}
\begin{minipage}[t]{0.3\textwidth}
    \vspace{0pt}
    \centering
    \includegraphics[width=3cm]{photos/Gabriel_Jordaan.jpg}
    
    \vspace{0.2cm}
    \textit{Gabriel Jordaan}
\end{minipage}
\hfill
\begin{minipage}[t]{0.65\textwidth}
    \vspace{0pt}
    \textbf{Major}: Electrical Engineering
    
    \textbf{Email}: gvanrijnjordaan@unr.edu

    \textbf{GitHub}: \url{https://github.com/ACertainArchangel}
    
    \vspace{0.3cm}
    \textbf{Role \& Contributions}:
    
    Project manager, hardware developer, enclosure design with Solidworks and 3D printing, music module, main.cpp, python utilities, documentation, and final report. Paid for parts and tools and is going to keep the Wetris unit after the project is over.
    
    \vspace{0.3cm}
    \textbf{About Me}:
    
    I am an electrical engineering student in the dual enrollment program at UNR, and as of fall 2025 I am 17 years old. I plan to graduate with my bachelors degree in 2027 and move on to a masters and PhD in robotics. Embedded systems and machine learning are my main interests in terms of engineering.
\end{minipage}

\vspace{0.5cm}

\subsection{Dieufainson Jean}
\begin{minipage}[t]{0.3\textwidth}
    \vspace{0pt}
    \centering
    \fbox{\parbox{3cm}{\centering\vspace{3cm}Photo\\Here}}
    
    \vspace{0.2cm}
    \textit{Dieufainson Jean}
\end{minipage}
\hfill
\begin{minipage}[t]{0.65\textwidth}
    \vspace{0pt}
    \textbf{Major}: Electrical Engineering
    
    \textbf{Email}: djean@unr.edu

    \textbf{GitHub}: \url{https://github.com/Sonylove5}
    
    \vspace{0.3cm}
    \textbf{Role \& Contributions}:
    
    Coding of the Potentiometer Module subsystem, a foundational UART communications layer, and the original Water Gun actuator control software.
    
    \vspace{0.3cm}
    \textbf{About Me}:
    
    Currently a Bachelor's student at the University of Nevada, Reno, my focus on power distribution is driven by my background as a licensed electrician and a Haitian immigrant, aiming to reduce global electricity costs and expand access to this essential resource.
\end{minipage}

\vspace{0.5cm}

\subsection{Jorge Robles}
\begin{minipage}[t]{0.3\textwidth}
    \vspace{0pt}
    \centering
    \fbox{\parbox{3cm}{\centering\vspace{3cm}Photo\\Here}}
    
    \vspace{0.2cm}
    \textit{[Jorge Robles}
\end{minipage}
\hfill
\begin{minipage}[t]{0.65\textwidth}
    \vspace{0pt}
    \textbf{Major}: [Electrical Engineering]
    
    \textbf{Email}: [jorgerobles@unr.edu]
    
    \textbf{GitHub}: \url{https://github.com/jrobles686}
    
    \vspace{0.3cm}
    \textbf{Role \& Contributions}:
    
    [Used GPIO methods to write code that did not implement any fuctions such as digitalWrite, digitalRead etc. Also added the interrupts, ISR, for the hardware buttons instead of polling. ]
    
    \vspace{0.3cm}
    \textbf{I am a first generation student working towards my bachelor’s degree in Electrical Engineering and am also minoring in Economic Policy. I enjoy learning how electronics communicate and operate in the real         world and am looking forward to working in the automotive industry post-graduation. Economics is my guilty pleasure and I enjoy using a different part of my brain to decipher how economic policies and  theories are         used today. }:
    
    [ I’ve worked in various fields throughout my career including the USPS, USFS, prestigious clubs and in the world of ophthalmology. I have plenty of exposure in customer satisfaction and am also fluent in Spanish. My     real passion is electronics, and I am pursuing my dream to become an Electrical Engineer. 
]
\end{minipage}

\vspace{0.5cm}

\subsection{Team Member 4}
\begin{minipage}[t]{0.3\textwidth}
    \vspace{0pt}
    \centering
    \fbox{\parbox{3cm}{\centering\vspace{3cm}Photo\\Here}}
    
    \vspace{0.2cm}
    \textit{[Member 4 Name]}
\end{minipage}
\hfill
\begin{minipage}[t]{0.65\textwidth}
    \vspace{0pt}
    \textbf{Major}: [Your Major]
    
    \textbf{Email}: [email@unr.edu]
    
    %\textbf{LinkedIn}: [linkedin.com/in/profile]
    
    \vspace{0.3cm}
    \textbf{Role \& Contributions}:
    
    [Describe your specific contributions to the project - hardware design, software modules, system integration, testing, etc.]
    
    \vspace{0.3cm}
    \textbf{About Me}:
    [Brief professional bio highlighting your skills, interests, career goals, relevant coursework, internships, projects, and what makes you stand out to potential employers.]
    
\end{minipage}

\vspace{0.5cm}

\clearpage

\section{Acknowledgments}

Thanks to our CPE 301 professor, Dr. Bashira Anima, for a great series of classes in fall 2025 and for being a good sport and letting us soak her with water!
\\\\
We would also like to acknowledge the open-source libraries that made this project possible:

\begin{itemize}
    \item LCDWIKI\_SPI and LCDWIKI\_GUI libraries - Created by the lcdwiki team (\url{https://github.com/lcdwiki/LCDWIKI_gui}), based on Adafruit GFX lib and Adafruit SPITFT lib, with init code from Rossum. Released under MIT license.
    \item Adafruit GFX and SPITFT libraries - For providing the foundational graphics and display functionality.
\end{itemize}

Thanks to Pastor David Minott for the idea of using a door lock actuator instead of a solenoid.\\

Thanks to Nikolai Nekrasov for writing the original Korobeiniki song in 1861. (He's dead and can't sue us for copyright infringement right?)

\clearpage

\section{Wetris Logo}

\begin{figure}[H]
\centering
\fbox{\parbox{0.8\textwidth}{\centering\includegraphics[width=0.8\textwidth]{photos/wetris_logo.png}}}
\caption{Wetris Logo}
\label{fig:logo}
\end{figure}

\end{document}
